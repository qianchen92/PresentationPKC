\begin{frame}{Rerandomizable encryption}
  \begin{block}{Setup and parameters}
    \begin{enumerate}
    \item $f,g \sample \G$ and $\hat{g}, \hat{h} \sample \hat{\G}$
    \item Choose random exponent $\alpha \sample \mathbb{Z}_p$ and $h = g^\alpha$.
    \item Choose Groth-Sahai proof witness-indistinguishable setting CRS $\PPP_{GS} = (\vec{u}_1, \vec{u}_2, \hat{\vec{u}}_1, \hat{\vec{u}}_2) \in \G^2 \times \G^2 \times \hat{\G}^2 \times \hat{\G}^2$.
    \item Define $\vec{v}_1 = (f, g, 1, 1, 1)$ and $\vec{v}_2 = (1, 1, 1, g, h)$.
    \item $(\SSK, \SVK) \gets LHSPS.\Setup$.
    \item Generate $\sigma_{\vec{v}_1}$ and $\sigma_{\vec{v}_2}$ (also is a proof of space $span(\vec{v}_1, \vec{v}_2)$).
    \item $\SK = \alpha \in \mathbb{Z}_p$ and
      \begin{align*}
        \PK &= (f,g,h,\PPP_{GS}, \hat{g}, \hat{h}) \in \G^{11} \times \hat{\G}^{16}
      \end{align*}
    \end{enumerate}
  \end{block}
\end{frame}


\begin{frame}{Rerandomizable encryption}

  %  \begin{block}{Details of the Encryption algorithm}
  %    \begin{itemize}
  %    \item Generate $\vec{C} = (C_0, C_1, C_2) = (MX^{\theta}, g_1^{\theta},  g_2^{\theta})$.
  %    \item $LHSPS.\KeyGen(\PPP) \to (\SSK, \SVK)$.
  %    \item $\Com(\ck, \SVK) \to (\com, \open)$.
  %    \item $\Prove(\PPP_{ABOProof}, x = (C_1, C_2), w = \theta, tag = \com) \to \pi$ of the fact $\exists \chi$ such that $(C_1, C_2) = (g_1^\chi, g_2^\chi)$.
  %    \item $LHSPS.\Sig(\SVK,((C_0,C_1,C_2, g),\pi)) \to \sigma_m$.
  %    \item $LHSPS.\Sig(\SVK,((X,g_1,g_2, 1),\pi)) \to \sigma_1$.
  %    \item $NIZK.\Prove(\PPP, m = (C_1, C_2), w = (\com,\pi)) \to \pi_{com}$
  %    \item $NIZK.\Prove(\PPP, m = (C_0,C_1,C_2,g), w = (\SVK, \sigma_m)) \to \pi_{\sigma_m}$
  %    \item $NIZK.\Prove(\PPP, m = (X,g_1,g_2,1), w = (\SVK, \sigma_1)) \to \pi_{\sigma_1}$
  %    \item Output the ciphertext $(\vec{C},\pi_{com}, \pi_{\sigma_m}, \pi_{\sigma_1}) \in \G^{39} \times \hat{\G}^{20}$.
  %    \end{itemize}
  % 
  %  \end{block}
  \begin{block}{Details of encryption}
  \begin{itemize}
    \pause
  \item Generate $\vec{C} = (c_0, c_1, c_2) = (f^{\theta},  g^{\theta}, M \cdot h^{\theta})$.
    \pause
  \item Set $b = 1$ and generate a Groth-Sahai proof for $b \in \{0,1\}$.
    \pause
  \item Define $\vec{\Theta} = (\Theta_0, \Theta_1, \Theta_2) = (c_0^b, c_1^b, c_2^b)$ and it's proof $\pi_{\vec{\Theta}}$.
    \pause
  \item Generate a LHSPS signature $\sigma_{\vec{v}}$ of the vector
    
    $\vec{v} = (c_0, c_1, 1, 1, 1) = (c_0^b, c_1^b, g^{1-b}, c_1^{1-b}, c_2^{1-b})$.
    \pause
  \item $\Prove(\PPP_{GS}, x = \SVK, w=(\sigma_{\vec{v}}, \vec{\Theta})) \to \pi_{enc}$ of the fact that $\sigma_{\vec{v}}$ is a valid signature of $\vec{v}$.
    \pause
  \item To enable the rerandomization we also need to set $(F, G, H) = (f^b, g^b, h^b)$ and the it's proof $\pi_{FGH}$.
    \pause
  \item Generate a signature $\sigma_{\vec{w}}$ of the vector $\vec{w} = (f, g, 1, 1, 1) = (f^b, g^b, 1, g^{1-b}, h^{1-b})$.
    \pause
  \item $\Prove(\PPP_{GS}, x = \SVK, w=(\sigma_{\vec{w}}, F, G, H)) \to \pi_{rand}$ of the fact that $\sigma_{\vec{w}}$ is a valid signature of $\vec{w}$.
    \pause
  \item Output the ciphertext $(c_1, c_2, \pi_{\vec{\Theta}}, \pi_{enc}, \pi_{FGH}, \pi_{rand})$.
  \end{itemize}
  \end{block}
\end{frame}

%\begin{frame}{Proof Intuition}
%  Recall: RCCA
% \begin{figure}
%    \begin{tikzpicture}
%      \node(A)[draw, thick, text width = 1.7cm, minimum height = 3cm] at (0,0) {Adversary};
%      \node(C)[draw, thick, text width = 1.7cm, minimum height = 3cm] at (4,0) {Challenger};
%      \node(D)[draw, thick, text width = 1.7cm, minimum height = 3cm] at (-4,0) {Decryption Oracle};
%      
%      \path([yshift = 1.4cm]A.east) edge[<-] node[above]{$\PK$} ([yshift = 1.4cm]C.west);
%      \path([yshift = 1cm]D.east) edge[<-] node[above]{$C_i$} ([yshift = 1cm]A.west);
%      \path([yshift = 0.6cm]D.east) edge[->] node[above]{$m_i$} ([yshift = 0.6cm]A.west);
%      \path([yshift = 0.2cm]A.east) edge[->] node[above]{$m_0, m_1$} ([yshift = 0.2cm]C.west);
%      
%      \path([yshift = -0.2cm]A.east) edge[<-] node[above]{$C_b$}([yshift = -0.2cm]C.west);
%      \path([yshift = -0.6cm]D.east) edge[<-] node[above]{Cond\footnote{$Dec(k_d, \{C_j\}) \not \in \{m_0,m_1\}$}} ([yshift = -0.6cm]A.west);
%      \path([yshift = -1cm]D.east) edge[->] node[above]{$m_i$} ([yshift = -1cm]A.west);
%      
%      \path([yshift = -1.4cm]A.east) edge[->] node[above]{$b'$}([yshift = -1.4cm]C.west);
%    \end{tikzpicture}
%    
%  \end{figure}
%  {\color{blue} Advantage:} $Adv(\Adv) = |\PR[b = b'] - \frac{1}{2}|$.
%\end{frame}


%\begin{frame}
%  \begin{columns}
%    \begin{column}{0.5\textwidth}
%    \tiny
%      \begin{figure}
%        \begin{tikzpicture}
%          \node(A)[draw, thick, text width = 1cm, minimum height = 4cm] at (0,0) {Adversary};
%          \node(C)[draw, thick, text width = 1cm, minimum height = 4cm] at (2,0) {Challenger};
%          \node(D)[draw, thick, text width = 1cm, minimum height = 4cm] at (-2,0) {Decryption Oracle};
%          
%          \path([yshift = 1.4cm]A.east) edge[<-] node[above]{$\PK$} ([yshift = 1.4cm]C.west);
%          \path([yshift = 1cm]D.east) edge[<-] node[above]{$C_i$} ([yshift = 1cm]A.west);
%          \path([yshift = 0.6cm]D.east) edge[->] node[above]{$m_i$} ([yshift = 0.6cm]A.west);
%          \path([yshift = 0.2cm]A.east) edge[->] node[above]{$m_0, m_1$} ([yshift = 0.2cm]C.west);
%          
%          \path([yshift = -0.2cm]A.east) edge[<-] node[above]{$C_b$}([yshift = -0.2cm]C.west);
%          \path([yshift = -0.6cm]D.east) edge[<-] node[above]{Cond\footnotemark} ([yshift = -0.6cm]A.west);
%          \path([yshift = -1cm]D.east) edge[->] node[above]{$m_i$} ([yshift = -1cm]A.west);
%          
%          \path([yshift = -1.4cm]A.east) edge[->] node[above]{$b'$}([yshift = -1.4cm]C.west);
%        \end{tikzpicture}
%      \end{figure}      
%      {\color{blue} Advantage:} $Adv(\Adv) = |\PR[b = b'] - \frac{1}{2}|$.
%    \end{column}
% 
%    \begin{column}{0.5\textwidth}
  %      \tiny
%  \begin{block}{Setup}
%  \begin{itemize}
%  \item \alt<5->{{\color{red}$h = f^xg^y$}}{$h = f^{\alpha}$.}
%  \item $CRS = (\vec{u}_1, \vec{u}_1, \hat{\vec{u}}_1, \hat{\vec{u}}_2)$ \alt<4->{\color{red}linealy dependent (soundness) $\vec{u}_1 = \vec{u}_2^\zeta$ and $\vec{u}_1 = \hat{\vec{u}}_2^{\hat{\zeta}}$.}{linearly independent (witness-indistinguishable).}
%  \end{itemize}
%  \end{block}
%      \begin{itemize}
%      \item Generate $\vec{C} = (c_0, c_1, c_2) = (f^{\theta},  g^{\theta}, M \cdot h^{\theta})$.
%      \item Set \alt<2->{{\color{red} $b = 0$}}{$b = 1$} and generate a Groth-Sahai proof for $b \in \{0,1\}$.
%      \item Define $\vec{\Theta} = (\Theta_0, \Theta_1, \Theta_2) = (c_0^b, c_1^b, c_2^b)$ and it's proof $\pi_{\vec{\Theta}}$.
%      \item \alt<3->{Generate a LHSPS signature $\sigma_{\vec{v}}$ {\color{red}using $\SSK$} of the vector
%        $\vec{v} = {\color{red}(1, 1, 1, c_1, c_2)} = (c_0^b, c_1^b, g^{1-b}, c_1^{1-b}, c_2^{1-b})$.}{Generate a LHSPS signature $\sigma_{\vec{v}}$ of the vector
%        $\vec{v} = (c_0, c_1, 1, 1, 1) = (c_0^b, c_1^b, g^{1-b}, c_1^{1-b}, c_2^{1-b})$.}
%      \item $\Prove(\PPP_{GS}, x = \SVK, w=(\sigma_{\vec{v}}, \vec{\Theta})) \to \pi_{enc}$ of the fact that $\sigma_{\vec{v}}$ is a valid signature of $\vec{v}$.
%      \item To enable the rerandomization we also need to set $(F, G, H) = (f^b, g^b, h^b)$ and the it's proof $\pi_{FGH}$.
%      \item Generate a signature $\sigma_{\vec{w}}$ of the vector $\vec{w} = (f, g, 1, 1, 1) = \alt<3->{\color{red}(f^{1-b}, g^{1-b}, 1, g^b, h^b)}{(f^b, g^b, 1, g^{1-b}, h^{1-b})}$.
%      \item $\Prove(\PPP_{GS}, x = \SVK, w=(\sigma_{\vec{w}}, F, G, H)) \to \pi_{rand}$ of the fact that $\sigma_{\vec{w}}$ is a valid signature of $\vec{w}$.
%      \item Output the ciphertext $(c_1, c_2, \pi_{\vec{\Theta}}, \pi_{enc}, \pi_{FGH}, \pi_{rand})$.
%      \end{itemize}
%    \end{column}
% 
%  \end{columns}
%  \footnotetext{$Dec(k_d, \{C_j\}) \not \in \{m_0,m_1\}$}
%\end{frame}
